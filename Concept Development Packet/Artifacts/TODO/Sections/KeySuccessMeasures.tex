
\chapter{Key Success Measures}
{
    \centering
   \resizebox{\textwidth}{!}
    {
    \begin{tabular}{>{\raggedright}p{3cm} >{\raggedright}p{2cm} >{\raggedright}p{2.3cm} >{\raggedright}p{2.3cm} >{\raggedright}p{2.2cm} >{\raggedright}p{1.2cm}  p{1.5cm} p{1.7cm}}
    
    \hline
   \textbf{Measure} & \textbf{Stretch Goal} & \textbf{Excellent Performance} & \textbf{Good Performance} & \textbf{Fair Performance} & \textbf{Lower Limit} & \textbf{Ideal} & \textbf{Upper Limit} \\
    \hline
    \hline
    
    
    
    Time to Train User &
    N/A &
    5 minutes &
    10 minutes &
    20 minutes &
    N/A &
    5 minutes &
    30 minutes \\
    
    \hline
    
    Percent Increase in Target Aquistion Time from Minimum Aquistion Time &
    10\% &
    20\% &
    30\% &
    40\% &
    N/A &
    0\% &
    50\% \\
    
    \hline
    
    Initial Setup Time &
     2 minutes &
     5 minutes &
     7 minutes &
     9 minutes &
     N/A &
     2 minutes &
     10 minutes\\
    
    \hline    
    
    \\
    
 \end{tabular}
 }
 }
 The key success measures shown above were chosen to help determine the desirability of the radar posititioning system (RPS). They will distinguish our design from a basic, functioning design. By achieving excellence in our key success measures we expect to exceed the customer's expectations. The team feels that these goals define our highest quality of work. Our key success measures account for the major flaws in IMSAR's current system. The user interface and setup time have caused the most issues for IMSAR and as such they drive our key succes measures. Reasoning for our defined measurements is given below.
 ~\\~\\
  The interface is currently barely usable and by running a survey and testing the interface with market representatives we intend to deliver an interface that does not require extensive training. It also takes a long time (10 minutes) to setup the RPS on site, due to the complexity of entering the data to control the RPS. We aim to reduce this by making it easier to enter the information and by improving the usage of non-volatile memory. The usage cases of IMSAR's radar units specify that every second matters when reacquiring the communication link. Mark was supportive of these key success measures and he participated in a team call in which we decided these key success measures (see NOTE-003).
  ~\\~\\
