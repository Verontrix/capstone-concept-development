\chapter{Introduction}

IMSAR is a local company that specializes in making compact radar systems more affordable and accessible for use in small air vehicles. One of the stationary ground systems currently produced and sold by IMSAR is a computer-controlled tilt and pan positioner that maintains a communication link with radar units installed on air vehicles. The operational situation for these units is to be in a stationary position 2 to 20 miles away from the target in-flight vehicle. The current design employs a tilt and pan positioner that has become obsolete. Our team has been tasked with replacing the tilt and pan positioner, installing an onboard computer, and creating control software necessary to maintain a communication link.  

~\\

Our team’s objective is to design, prototype, and test a Radar Positioning System (RPS) that can track in-flight vehicles while maintaining visual contact by March 30, 2020 for under \$1,500 and 1,500 man-hours. The overall performance of the system will be evaluated based on the following key success measures: time to train the user, percent increase in target acquisition time from minimum acquisition time, and the initial setup time. Ultimately the amount of time that the aircraft is within the field of view (i.e. the percent of time that a communication link is maintained) is the most important criteria. IMSAR’s current system maintains a communication link 100\% of the time under normal use conditions. Thus, we have determined that it is necessary for our end product to also maintain a communication link 100\% of the time. Because this is a binary success or failure, it is not included in the key success measures, however, it is the most important criteria for the system. Full system requirements can be found in the requirements matrix (RM-001) attached to this document. 

 ~\\

During the concept development stage three primary subsystems were defined i.e. the user interface, the system architecture, and the tracking method. Concepts for each of these subsystems were developed, and decisions on which concepts to pursue were made.